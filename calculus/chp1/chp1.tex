\documentclass[a4paper,14pt]{report}
\usepackage{fullpage}
%///////////////////////////////////////////////////////////
%///////////////////////////////////////////////////////////
\usepackage{imakeidx}


%///////////////////////////////////////////////////////////
%///////////////////////////////////////////////////////////
%///////////////////////////////////////////////////////////
\usepackage{extsizes} % Возможность сделать 14-й шрифт

%///////////////////////////////////////////////////////////
%%% Работа с русским языком
\usepackage{cmap}					% поиск в PDF
%\usepackage{mathtext} 				% русские буквы в фомулах
\usepackage[T2A]{fontenc}			% кодировка
\usepackage[utf8]{inputenc}			% кодировка исходного текста
\usepackage[english,russian]{babel}	% локализация и переносы
\usepackage{lipsum}                 % use \lipsum[1-4] \lipsum[] \lipum[2]
%///////////////////////////////////////////////////////////
% для красивых таблиц
\usepackage{booktabs}

%///////////////////////////////////////////////////////////
% для кастомизации списков
\usepackage{enumitem}
%\begin{itemize}[noitemsep,nolistsep] %
    
%///////////////////////////////////////////////////////////
%%% Дополнительная работа с математикой
\usepackage{amsmath,amsfonts,amssymb,amsthm,mathtools} % AMS
\usepackage{icomma} % "Умная" запятая: $0,2$ --- число, $0, 2$ --- перечисление

\DeclarePairedDelimiter\abs{\lvert}{\rvert}%
\DeclarePairedDelimiter\norm{\lVert}{\rVert}%

\makeatletter
\let\oldabs\abs
\def\abs{\@ifstar{\oldabs}{\oldabs*}}
%
\let\oldnorm\norm
\def\norm{\@ifstar{\oldnorm}{\oldnorm*}}
\makeatother

%///////////////////////////////////////////////////////////   
% активные ссылки
\usepackage[unicode]{hyperref}
\usepackage{cleveref}

% для кастомных ссылок 
\newcommand\myref[1]{\hyperref[#1]{#1}}

% цвет ссылок
\hypersetup{
    colorlinks=true,
    linkcolor=blue,
    filecolor=magenta,      
    urlcolor=cyan,
    pdftitle={calculus},
    pdfpagemode=FullScreen,
}

\urlstyle{same}

%///////////////////////////////////////////////////////////
%% Номера формул
\mathtoolsset{showonlyrefs=false} % Показывать номера только у тех формул, на которые есть \eqref{} в тексте.
%\usepackage{leqno} % Нумерация выражений слева

%///////////////////////////////////////////////////////////
%% Шрифты
\usepackage{euscript}	 % Шрифт Евклид
\usepackage{mathrsfs}    % Красивый матшрифт

%///////////////////////////////////////////////////////////   
\usepackage{setspace} % Интерлиньяж
%\onehalfspacing % Интерлиньяж 1.5
%\doublespacing % Интерлиньяж 2
%\singlespacing % Интерлиньяж 1

%///////////////////////////////////////////////////////////    
%% Перенос знаков в формулах (по Львовскому)
\newcommand*{\hm}[1]{#1\nobreak\discretionary{}
    {\hbox{$\mathsurround=0pt #1$}}{}}

%///////////////////////////////////////////////////////////
\usepackage{graphicx}
% \graphicspath{ {./prints/} }


% bigcdot ( \cdot < \bigcdot < \bullet )
\makeatletter
\newcommand*\bigcdot{\mathpalette\bigcdot@{.5}}
\newcommand*\bigcdot@[2]{\mathbin{\vcenter{\hbox{\scalebox{#2}{$\m@th#1\bullet$}}}}}
\makeatother

%/////////////////////////////////////////////////////////// 
% Для обозначения множеств
\newcommand{\setR}{\ensuremath{\mathbb R}}
\newcommand{\setQ}{\ensuremath{\mathbb Q}}
\newcommand{\setZ}{\ensuremath{\mathbb Z}}
\newcommand{\setN}{\ensuremath{\mathbb N}}
% Для обозначения разбиения отрезка
\newcommand{\partit}{\ensuremath{\mathbb P}}
% Для обозначения отрезка $[a,b]$
\newcommand{\ab}{\ensuremath{[a,b]}}
% Для ссылки на страницу
\newcommand{\str}[1]{%
на стр. \pageref{#1}}
% Больше или равно
\renewcommand{\geq}{\geqslant}
\renewcommand{\leq}{\leqslant}
% Меньше или равно


%///////////////////////////////////////////////////////////
% Библиография
\usepackage{cite}
\usepackage{csquotes}
%\renewcommand{\refname}{Список источников}  % По умолчанию "Список литературы" (article)
%\renewcommand{\bibname}{Литература}  % По умолчанию "Литература" (book и report)

%///////////////////////////////////////////////////////////
% squared itemize 
\newcounter{boxlblcounter}  
\newcommand{\makeboxlabel}[1]{\fbox{#1.}\hfill}% \hfill fills the label box
\newenvironment{boxlabel}
{\begin{list}
        {\arabic{boxlblcounter}}
        {
            \usecounter{boxlblcounter}
            \setlength{\labelwidth}{3em}
            \setlength{\labelsep}{0em}
            \setlength{\itemsep}{2pt}
            \setlength{\leftmargin}{1.5cm}
            \setlength{\rightmargin}{2cm}
            \setlength{\itemindent}{0em} 
            \let\makelabel=\makeboxlabel
        }
    }
{\end{list}}

%///////////////////////////////////////////////////////////
% Счетчик для теорем
\newcounter{counterTheorem}[section]
\setcounter{counterTheorem}{0}

% Счетчик для лемм
\newcounter{counterLlemma}[section]

% Счетчик для доказательств
\newcounter{counterProof}[section]

% Счетчик для определений
\newcounter{counterDef}[section]
\setcounter{counterDef}{1}
% Счетчик для утверждений
\newcounter{counterStatement}[section]

%///////////////////////////////////////////////////////////
% Так можно задавать свою длину
\newlength{\mylength}
\settowidth{\mylength}{some text}
% \begin{minipage}{\mylength}
    % {\textit{lipsum[1]}
        % }
    % \end{minipage}

%///////////////////////////////////////////////////////////
\newtheorem{theorem}{Теорема}
\newtheorem{lemma}{Лемма}
\newtheorem{definition}{Определение}
\newtheorem{example}{Пример}
\newtheorem{consequence}{Следствие}
\newtheorem{statement}{Утверждение}

%///////////////////////////////////////////////////////////
%///////////////////////////////////////////////////////////
%///////////////////////////////////////////////////////////
%///////////////////////////////////////////////////////////
%///////////////////////////////////////////////////////////
%///////////////////////////////////////////////////////////
%///////////////////////////////////////////////////////////
\newcommand{\rom}[1]{\MakeUppercase{\romannumeral #1}}
%///////////////////////////////////////////////////////////
%///////////////////////////////////////////////////////////
\newenvironment{_def}[1][]{%
\addtocounter{counterDef}{1}
\textbf{Определение \arabic{counterDef}.}
}{\vspace{0.2cm}}
%///////////////////////////////////////////////////////////
%///////////////////////////////////////////////////////////
%///////////////////////////////////////////////////////////
% \usepackage{fullpage}
\title{Математический анализ 1} 
\author{Rustem Sirazetdinov}
\date{4/07/2022}

\makeindex
\begin{document}
    % \maketitle   
    % \tableofcontents
    
    
    \chapter*{Вступление}
    \addcontentsline{toc}{chapter}{Вступление}
    
    Здесь будет вступление? Может быть, если я его придумаю, конечно.
    
    \chapter{Действительные числа}    
    В математическом анализе есть основная неарифметическая операция предельного перехода, с которой мы будем встречаться сплошь и рядом. В основе этой операции лежит свойство полноты(непрерывности) числового множества, на котором определена функция. А поскольку особое внимание в анализе уделяется числовым функциям, то лучшим вариантом будет начать изучение "особых"\ множеств с действительных чисел, поэтому цель этой главы: дать точное определение вещественных чисел и обратить внимание на их свойства. 
    
    \section{Аксиоматика и некоторые свойства}
    
    \subsection{Определение действительных чисел}\label{Определение}
    
    \begin{_def}\index{Действительные числа}
        Множество \setR\ называется множеством \textbf{действительных чисел}, а его элементы - действительными числами, если выполнены следующий комплекс аксиом.
    \end{_def}

    (\rom{1}) \textsc{Аксиомы сложения.} \par
    Определена операция сложения $$ +: \setR \times \setR \rightarrow \setR, $$ сопоставляющая каждой паре $ (x, y) $ элементов из \setR\ некоторый элемент $ x+y \in \setR $, называемый \textit{суммой} $ x $ и $ y $. При этом выполнены условия:
    
    \begin{enumerate}[noitemsep,nolistsep]
        \item Существование \textit{нейтрального}\index{Аксиома!--существование нейтрального!--относительно сложения} элемента \[ x + 0 = 0 + x = x\ \forall x \in \setR \]
        \item Существование \textit{противоположного}\textit{нейтрального}\index{Аксиома!--существование противоположного} элемента $ -x \in \setR $ такого, что \[ x + (-x)  = (-x) + x = 0\ \forall x \in \setR\]
        \item Операция сложения ассоциативна\index{Аксиома!--ассоциативность!--сложения} \[ x + (y + z) = (x + y) + z\ \forall x, y, z \in \setR \]
        \item Операция сложения коммутативна\index{Аксиома!--коммутативность!--сложения} \[ x + y = y + x\ \forall x, y \in \setR \]
    \end{enumerate}
    
    \noindent Другими словами множество \setR\ это \textit{аддитивная абелева\footnote{т.е. коммутативная} группа}.\index{Группа!--аддитивная}\index{Группа!--абелева}
   	
   	\bigskip
   	(\rom{2}) \textsc{Аксиомы умножения.} \par
   	Определена операция умножения \[ \bigcdot\ : \setR \times \setR \rightarrow \setR, \] сопоставляющая каждой паре $ (x, y) $ элементов из \setR\ некоторый элемент $ x\ \bigcdot y \in \setR $, называемый \textit{произведением} $ x $ и $ y $. При этом выполнены условия:
   	
   \begin{enumerate}[noitemsep,nolistsep]
		\item Существование \textit{нейтрального}\index{Аксиома!--существование нейтрального!--относительно умножения} элемента \[ x \cdot 1 = 1 \cdot x = x\ \forall x \in \setR \backslash{0} \]
		\item Существование \textit{обратного}\index{Аксиома!--существование обратного} элемента $ x^{-1} \in \setR $ такого, что \[ x \cdot x^{-1}  = x^{-1} \cdot x = 1\ \forall x \in \setR \backslash 0 \]
		\item Операция сложения ассоциативна\index{Аксиома!--ассоциативность!--умножения} \[ x \cdot (y \cdot z) = (x \cdot y) \cdot z\ \forall x, y, z \in \setR \]
		\item Операция сложения коммутативна\index{Аксиома!--коммутативность!--умножения} \[ x \cdot y = y \cdot x\  \forall x, y \in \setR \]
   \end{enumerate}
   
   \noindent Таким образом относительно умножения множество \setR $ \backslash 0 $ мультипликативная группа\index{Группа!--мультипликативная}.  
   
   \vspace{0.3cm}
   (\rom{1}, \rom{2}) \textsc{Связь сложения и умножения.} \par
   Операции сложения и умножения связаны законом \textit{дистрибутивности}\index{Дистрибутивность}. Умножение дистрибутивно по отношению к сложению \[ (x + y)z = xz + yz\ \forall x, y, z \in \setR \]
   
   \noindent \textbf{($ \ast $)} Если на каком-то множестве $ G $ действуют две бинарные операции, удовлетворяющие всем перечисленным аксиомам, то $ G $ называется \textit{алгебраическим полем}\index{Поле!--алгебраическое} или просто \textit{полем}\index{Поле}.
   
   \vspace{0.3cm}
   (\rom{3}) \textsc{Аксиомы порядка.} \par
   Между элементами \setR\ имеется отношение нестрогого линейного порядка $ \leq $, т.е. для всех элементов мы знаем результат\footnotemark{} выражения $ x \leq y $. При этом выполнены условия:
   
    \begin{enumerate}
    	\item[0.] $ (x \leq x)\ \forall x \in \setR $ - \textit{рефлексивность}\index{Отношение!--рефлексивное}
    	\item $ (x \leq y) \wedge (y \leq x) \Rightarrow (x = y) $ - \textit{симметричность}\index{Отношение!--симметричное}
    	\item $ (x \leq y) \wedge (y \leq z) \Rightarrow (x \leq z) $ - \textit{транзитивность}\index{Отношение!--транзитивное}
    	\item $ (x \leq y) \vee (y \leq x)\ \forall x, y \in \setR $
	\end{enumerate}
   
    В таком случае отношение $ \leq $ называется отношением неравенства\index{Отношение!--неравенства}. 
    
    \noindent \textbf{($ \ast $)} Если множество удовлетворяет аксиомам 0,1,2, то говорят, что это множество \textit{частично упорядочено}\index{Порядок!--частичный}. Аксиома 3 говорит, что любые два элемента множества сравнимы, если же помимо остальных выполнена и она, то говорят, что множество \textit{линейно упорядочено}\index{Порядок!--линейный}.
    
    \noindent \textbf{($ \ast $)} На самом деле отношение порядка\index{Отношение!--порядка} будет играть важную роль, ведь отношение порядка позволяет нам сравнивать элементы множества, а это в свою очередь позволит нам ввести и использовать понятие монотонности. 
    
    \footnotetext{\textsc{true} or \textsc{false}}
    
    \vspace{0.3cm}
    (\rom{1}, \rom{3}) \textsc{Связь сложения и порядка} \par
    \[ (x \leq y) \Rightarrow (x + z \leq y + z)\ \forall x, y, z \in \setR \]
    
    (\rom{2}, \rom{3}) \textsc{Связь умножения и порядка} \par
    \[ (0 \leq x) \wedge (0 \leq y) \Rightarrow (0 \leq x \cdot y) \]
    
    \noindent Теперь наконец-то перейдем к тому, ради чего все затевалось.
    
    (\rom{4}) \textsc{Аксиома полноты(непрерывности)} \par
   
   
   
   
   
   
   
   
   
   
   
   
   
   
   
   
   
   
   
   
   
   
   
   
   
   
   
\printindex
\end{document}
    